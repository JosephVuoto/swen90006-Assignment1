\documentclass{article}
\usepackage{fancyhdr}
\usepackage{graphicx}
\pagestyle{fancy}
\usepackage{geometry}
\usepackage{listings}
\usepackage{longtable}
\usepackage{enumitem}
\usepackage{color}
\usepackage{fontspec}
\geometry{a4paper, scale=0.75}
\lhead{Yangzhe Xie}
\chead{SWEN90006: Assignment 1}
\rhead{1029787}
\renewcommand{\headrulewidth}{0.4pt}
\renewcommand{\headwidth}{\textwidth}
\definecolor{keywordcolor}{rgb}{0.8,0.1,0.5}
\definecolor{webgreen}{rgb}{0,.5,0}
\definecolor{bgcolor}{rgb}{0.92,0.92,0.92}
\lstset{language=[AspectJ]Java,
    %backgroundcolor=\color{bgcolor}, 设置背景色
    basicstyle=\small,
    commentstyle=\color{blue} \textit, 
    showstringspaces=false,
    captionpos=b,
    xleftmargin=2em,
    xrightmargin=2em, 
    aboveskip=1em
    %numbers=left,
    %numberstyle=\small
}

\title{SWEN90006: Assignment 1}
\author{Name: Yangzhe Xie\\Student number: 1029787\\Email: yangzhe.xie@student.unimelb.edu.au}

\date{\today}
\begin{document}

\maketitle
\thispagestyle{fancy}

\section{Task 1}
\subsection{Test template trees}
Figure 1 - 4 shows the test template trees for the API addUser, loginUser, updateDetails, and retrieveDetails respectively.
\begin{figure}[hbt!]        
\center{\includegraphics[width=17cm]  {addUserNew.png}}        
\caption{\label{1} Test template tree for addUser()}      
\end{figure}
\begin{figure}[hbt!]        
\center{\includegraphics[width=15cm]  {loginUserNew.png}}        
\caption{\label{1} Test template tree for loginUser()}      
\end{figure}
\begin{figure}[hbt!]        
\center{\includegraphics[width=11cm]  {updateDetailsNew.png}}        
\caption{\label{1} Test template tree for updateDetails()}      
\end{figure}
\begin{figure}[t!]        
\center{\includegraphics[width=12cm]  {retrieveDetailsNew.png}}        
\caption{\label{1} Test template tree for retrieveDetails()}      
\end{figure}

\subsection{Do your set of equivalence classes cover the input space?}
My set of equivalence classes cover the input space. The reasons are as follows:
\begin{itemize}
\item [1)] All leaf nodes are divided strictly and carefully, so that they do not overlap with other leaf.
\item [2)] The collection of the set of each sibling node covers all the cases of their parent node.
\item [3)]
If two variables are independent of each other, then the subtree of one variable can be added to a leaf node of the other variable. In this case, all the nodes add up to cover all situations.
\item [4)]
As part of your input domain, the instance variables should also be considered. Note that all of these variables are collections, so according to guideline 4, we should follow the zero-one-many rule. But in this particular case, we just care about whether the collection contains some values. So I combined the two cases (number of elements equals 1 and greater than 1) into one (greater than 0), which does not affect the results of the tests.
\end{itemize}

\section{Test cases associated with equivalence classes}
\subsection{addUser}

\begin{longtable}{|p{2cm}|p{7cm}|p{5cm}|}
\caption{Test cases for addUser}\\
\hline 
ID&Test case&Expected output\\
\hline  
EC1&paraphrases = \{\}, passbookUsername = "abc", paraphrase = "12345aA"&WeakPassphraseException\\
\hline
EC2&paraphrases = \{\}, passbookUsername = "abc", paraphrase = "1234567A"&WeakPassphraseException\\
\hline
EC3&paraphrases = \{\}, passbookUsername = "abc", paraphrase = "1234567a"&WeakPassphraseException\\
\hline
EC4&paraphrases = \{\}, passbookUsername = "abc", paraphrase = "123456aA"&-\\
\hline
EC5&paraphrases = \{\}, passbookUsername = "abc", paraphrase = "abcdABCD"&WeakPassphraseException\\
\hline
EC6&paraphrases = \{"abcd":"123456aA"\}, passbookUsername = "abcd", paraphrase = "123456aA"&DuplicateUserException\\
\hline
EC7&paraphrases = \{"abcd":"123456aA"\}, passbookUsername = "abc", paraphrase = "123456aA"&-\\
\hline
\end{longtable}

\subsection{loginUser}

\begin{longtable}{|p{2cm}|p{7cm}|p{5cm}|}
\caption{Test cases for loginUser}\\
\hline 
ID&Test case&Expected output\\
\hline  
EC1&paraphrases = \{\}, sessionIDs = \{\}, userIDs = \{\}, passbookUsername = "abc", paraphrase = "123456aA"&NoSuchUserException\\
\hline
EC2&paraphrases = \{"abc":"123456aA"\}, sessionIDs = \{\}, userIDs = \{\} passbookUsername = "abc", paraphrase = "123456aB"&IncorrectPassphraseException\\
\hline
EC3&paraphrases = \{"abc":"123456aA"\}, sessionIDs = \{\}, userIDs = \{\} passbookUsername = "abc", paraphrase = "123456aA"&...\\
\hline
EC4&paraphrases = \{"abc":"123456aA"\}, sessionIDs = \{\}, userIDs = \{123:"def"\} passbookUsername = "abc", paraphrase = "123456aA"&...\\
\hline
EC5&paraphrases = \{"abc":"123456aA"\}, sessionIDs = \{"def":123\}, userIDs = \{\} passbookUsername = "abc", paraphrase = "123456aA"&...\\
\hline
EC6&paraphrases = \{"abc":"123456aA"\}, sessionIDs = \{"abc":123\}, userIDs = \{\} passbookUsername = "abc", paraphrase = "123456aA"&AlreadyLoggedInException\\
\hline
EC7&paraphrases = \{"abc":"123456aA"\}, sessionIDs = \{\}, userIDs = \{\} passbookUsername = "abcd", paraphrase = "123456aA"&NoSuchUserException\\
\hline
\end{longtable}


\subsection{updateDetails}
\begin{longtable}{|p{2cm}|p{7cm}|p{5cm}|}
\caption{Test cases for updateUserDetails}\\
\hline 
ID&Test case&Expected output\\
\hline  
EC1&userIDs = \{\}, sessionID = 123, url = "http://test.com", urlUsername = "123", urlPassword = "123"&InvalidSessionIDException\\
\hline
EC2&userIDs = \{123:"abc"\}, sessionID = 456, url = "http://test.com", urlUsername = "123", urlPassword = "123"&InvalidSessionIDException\\
\hline
EC3&userIDs = \{123:"abc"\}, sessionID = 123, url = "http://test.com", urlUsername = "123", urlPassword = "123"&-\\
\hline
EC4&userIDs = \{123:"abc"\}, sessionID = 123, url = "https://test.com", urlUsername = null, urlPassword = "123"&-\\
\hline
EC5&userIDs = \{123:"abc"\}, sessionID = 123, url = "https://test.com", urlUsername = "123", urlPassword = null&-\\
\hline
EC6&userIDs = \{123:"abc"\}, sessionID = 123, url = "https://test.com", urlUsername = "123", urlPassword = "123"&-\\
\hline
EC7&userIDs = \{123:"abc"\}, sessionID = 123, url = "ftp://test.com", urlUsername = "123", urlPassword = "123"&MalformedURLException\\
\hline
\end{longtable}
\subsection{retrieveDetails}

%---------------------------------------------------------------------------------------
\section{Boundary-value analysis}
\subsection{addUser}

\begin{longtable}{|p{2cm}|p{7cm}|p{5cm}|}
\caption{Test cases for addUser}\\
\hline 
EC&Test case&Remark\\
\hline  
EC1&paraphrases = \{\}, passbookUsername = "abc", paraphrase = "12345aA"&off point for paraphrase.length() < 8\\
\hline
EC2&paraphrases = \{\}, passbookUsername = "abc", paraphrase = "123456`A"&off point for S[i] not in ‘a’ - ‘z’, for every 0 <= i <= length[paraphrase]-1\\
\hline
EC2&paraphrases = \{\}, passbookUsername = "abc", paraphrase = "123456\{A"&off point for S[i] not  in ‘a’ - ‘z’, for every 0 <= i <= length[paraphrase]-1\\
\hline
EC3&paraphrases = \{\}, passbookUsername = "abc", paraphrase = "123456@a"&off point for S[i] not in ‘A’ - ‘Z’, for every 0 <= i <= length[paraphrase]-1\\
\hline
EC3&paraphrases = \{\}, passbookUsername = "abc", paraphrase = "123456[a"&off point for S[i] not  in ‘A’ - ‘Z’, for every 0 <= i <= length[paraphrase]-1\\
\hline
EC4&paraphrases = \{\}, passbookUsername = "abc", paraphrase = "234567nA"&on point for S[i] in ‘A’ - ‘Z’, for some 0 <= i <= length[paraphrase]-1\\
\hline
EC4&paraphrases = \{\}, passbookUsername = "abc", paraphrase = "234567nZ"&on point for S[i] in ‘A’ - ‘Z’, for some 0 <= i <= length[paraphrase]-1\\
\hline
EC4&paraphrases = \{\}, passbookUsername = "abc", paraphrase = "234567Na"&on point for S[i] in ‘a’ - ‘z’, for some 0 <= i <= length[paraphrase]-1\\
\hline
EC4&paraphrases = \{\}, passbookUsername = "abc", paraphrase = "234567Nz"&on point for S[i] in ‘a’ - ‘z’, for some 0 <= i <= length[paraphrase]-1\\
\hline
EC4&paraphrases = \{\}, passbookUsername = "abc", paraphrase = "abcdABC0"&on point for S[i] in ‘0’ - ‘9’, for some 0 <= i <= length[paraphrase]-1\\
\hline
EC4&paraphrases = \{\}, passbookUsername = "abc", paraphrase = "abcdABC9"&on point for S[i] in ‘0’ - ‘9’, for some 0 <= i <= length[paraphrase]-1\\
\hline
EC5&paraphrases = \{\}, passbookUsername = "abc", paraphrase = "abcdABC/"&off point for S[i] not in ‘0’ - ‘9’, for every 0 <= i <= length[paraphrase]-1\\
\hline
EC5&paraphrases = \{\}, passbookUsername = "abc", paraphrase = "abcdABC:"&off point for S[i] not in ‘0’ - ‘9’, for every 0 <= i <= length[paraphrase]-1\\
\hline
EC6&paraphrases = \{"abcd":"123456aA"\}, passbookUsername = "abcd", paraphrase = "123456aA"&on point for passbookUsername in passphrases.keys()\\
\hline
EC7&paraphrases = \{"abcd":"123456aA"\}, passbookUsername = "abc", paraphrase = "123456aA"&on point for passbookUsername not in passphrases.keys()\\
\hline
\end{longtable}

\subsection{loginUser}

\begin{longtable}{|p{2cm}|p{7cm}|p{5cm}|}
\caption{Test cases for loginUser}\\
\hline 
ID&Test case&Expected output\\
\hline  
EC1&paraphrases = \{\}, sessionIDs = \{\}, userIDs = \{\}, passbookUsername = "abc", paraphrase = "123456aA"&NoSuchUserException\\
\hline
EC2&paraphrases = \{"abc":"123456aA"\}, sessionIDs = \{\}, userIDs = \{\} passbookUsername = "abc", paraphrase = "123456aB"&IncorrectPassphraseException\\
\hline
EC3&paraphrases = \{"abc":"123456aA"\}, sessionIDs = \{\}, userIDs = \{\} passbookUsername = "abc", paraphrase = "123456aA"&...\\
\hline
EC4&paraphrases = \{"abc":"123456aA"\}, sessionIDs = \{\}, userIDs = \{123:"def"\} passbookUsername = "abc", paraphrase = "123456aA"&...\\
\hline
EC5&paraphrases = \{"abc":"123456aA"\}, sessionIDs = \{"def":123\}, userIDs = \{\} passbookUsername = "abc", paraphrase = "123456aA"&...\\
\hline
EC6&paraphrases = \{"abc":"123456aA"\}, sessionIDs = \{"abc":123\}, userIDs = \{\} passbookUsername = "abc", paraphrase = "123456aA"&AlreadyLoggedInException\\
\hline
EC7&paraphrases = \{"abc":"123456aA"\}, sessionIDs = \{\}, userIDs = \{\} passbookUsername = "abcd", paraphrase = "123456aA"&NoSuchUserException\\
\hline
\end{longtable}


\subsection{updateDetails}
\begin{longtable}{|p{2cm}|p{7cm}|p{5cm}|}
\caption{Test cases for updateUserDetails}\\
\hline 
ID&Test case&Expected output\\
\hline  
EC1&userIDs = \{\}, sessionID = 123, url = "http://test.com", urlUsername = "123", urlPassword = "123"&InvalidSessionIDException\\
\hline
EC2&userIDs = \{123:"abc"\}, sessionID = 456, url = "http://test.com", urlUsername = "123", urlPassword = "123"&InvalidSessionIDException\\
\hline
EC3&userIDs = \{123:"abc"\}, sessionID = 123, url = "http://test.com", urlUsername = "123", urlPassword = "123"&-\\
\hline
EC4&userIDs = \{123:"abc"\}, sessionID = 123, url = "https://test.com", urlUsername = null, urlPassword = "123"&-\\
\hline
EC5&userIDs = \{123:"abc"\}, sessionID = 123, url = "https://test.com", urlUsername = "123", urlPassword = null&-\\
\hline
EC6&userIDs = \{123:"abc"\}, sessionID = 123, url = "https://test.com", urlUsername = "123", urlPassword = "123"&-\\
\hline
EC7&userIDs = \{123:"abc"\}, sessionID = 123, url = "ftp://test.com", urlUsername = "123", urlPassword = "123"&MalformedURLException\\
\hline
\end{longtable}
\subsection{retrieveDetails}

\enddocument